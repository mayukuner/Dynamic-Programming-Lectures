\documentclass[hyperref={unicode=true}]{beamer}
\usepackage{multirow}
\usepackage{minted}
\usepackage{amsmath} % used for boldsymbol.
\renewcommand{\vec}[1]{\boldsymbol{#1}} % Uncomment for BOLD vectors.
\usepackage[slantfont,boldfont]{xeCJK}
\setCJKmainfont{SimSun}
\setCJKmonofont{SimHei}
\usetheme{Darmstadt}
\usecolortheme{beaver}
\usepackage{booktabs}

\input zhwinfonts
\begin{document}
\setbeamertemplate{caption}[numbered]
\renewcommand\figurename{图}
\renewcommand\tablename{表}
\renewcommand\contentsname{\centering 目录}

%%------------------------------------------
\title{动态规划进阶(一)}
\subtitle{}
\author{马玉坤}
\institute{哈尔滨工业大学计算机科学与技术学院}
\date{2017年8月17日}
%%------------------------------------------

\begin{frame}\titlepage\end{frame}

  \section{区间动态规划}
  \begin{frame}\frametitle{石子合并}
    \framesubtitle{第九次CCF计算机软件能力认证}
    \begin{block}{题目}
      设有n堆石子排成一排,其编号为1,2,3,…,n ($n\leq500$)。每堆石子有一定的数量。现要将n堆石子合并成为一堆。归并的过程只能每次将相邻的两堆石子堆成一堆,合并的代价是两堆石子的石子数量和。这样n堆石子经过n-1次归并后成为一堆。\\
      找出将n堆石子合并成一堆石子的最小总代价。
    \end{block}
  \end{frame}

  \begin{frame}\frametitle{石子合并 (Cont'd)}
    \framesubtitle{第九次CCF计算机软件能力认证}
    \begin{exampleblock}{竟然可以过的解法}
      设$dp[i][j]$表示第i堆到第j堆石子合并成一堆所需要的最小代价,那么我们有:
      \[dp[i][j] = \min_{i < k \leq j}dp[i][k-1] + dp[k][j] + w[i][j]\]
      其中$w[i][j]$表示第i堆到第j堆石子的石子数目和。
    \end{exampleblock}
  \end{frame}

  \begin{frame}\frametitle{石子合并 (Cont'd)}
    \framesubtitle{第九次CCF计算机软件能力认证}
    \begin{alertblock}{四边形不等式}
    \end{alertblock}
  \end{frame}
  \begin{frame}\frametitle{石子合并 (Cont'd)}
    \framesubtitle{第九次CCF计算机软件能力认证}
    \begin{exampleblock}{更快的解法}
    \end{exampleblock}
  \end{frame}

  \begin{frame}\frametitle{Do Geese See God?}
    \framesubtitle{Asia Tsukuba Regional Contest 2015 G}
    \begin{block}{题目}
      给定一个字符串,添加最少的字符使得这个字符串变成回文串。\\
      \pause{}还没完。\\
      在所有这样的最短的回文串中,找到字典序第$k$小的并输出。
      字符串长度$\leq 2000$,$k \leq 10^{18}$。
    \end{block}
  \end{frame}
  \begin{frame}\frametitle{Do Geese See God? (Cont'd)}
    \framesubtitle{Asia Tsukuba Regional Contest 2015 G}

  \end{frame}

\end{document}
