\documentclass[hyperref={unicode=true}]{beamer}
\usepackage{multirow}
\usepackage{minted}
\usepackage{amsmath} % used for boldsymbol.
\renewcommand{\vec}[1]{\boldsymbol{#1}} % Uncomment for BOLD vectors.
\usepackage[slantfont,boldfont]{xeCJK}
\setCJKmainfont{SimSun}
\setCJKmonofont{FangSong}
\setCJKsansfont{SimHei}
\usetheme{Darmstadt}
\usecolortheme{beaver}
\usepackage{booktabs}
\usepackage{tikz}
\setbeamertemplate{theorems}[numbered]
\newtheorem{mytht}{\bf 定理}
\theoremstyle{definition}
\newtheorem{mydef}[]{\bf 定义}
\theoremstyle{proof}
\newtheorem{myprf}[]{\bf 证明}
\usepackage{ulem}
\usepackage[T1]{fontenc}
\usepackage{inconsolata}
\makeatletter
% there's no italic/slanted for Inconsolata
\@namedef{T1/zi4/m/it}{<->ssub*zi4/m/n}
\@namedef{T1/zi4/b/it}{<->ssub*zi4/b/n}
\@namedef{T1/zi4/bx/it}{<->ssub*zi4/b/n}
\@namedef{T1/zi4/m/sl}{<->ssub*zi4/m/n}
\@namedef{T1/zi4/b/sl}{<->ssub*zi4/b/n}
\@namedef{T1/zi4/bx/sl}{<->ssub*zi4/b/n}
\makeatother

\input zhwinfonts
\begin{document}
\setbeamertemplate{caption}[numbered]
\renewcommand\figurename{图}
\renewcommand\tablename{表}
\renewcommand\contentsname{\centering 目录}


%%------------------------------------------
\title{动态规划进阶(一)练习题题解}
\author{马玉坤}
\institute{哈尔滨工业大学计算机科学与技术学院}
\date{2017年8月17日}
%%------------------------------------------

\begin{frame}\titlepage\end{frame}

\begin{frame}{Huffman's Greed}
  \framesubtitle{POJ 1784}
  先求出$w[i][j]$,即数字$>a[i-1]$且$<a[i+1]$的概率。然后就有
  \[dp[i][j] = \min_{k=i+1}^{j-1}dp[i][k-1] + dp[k+1][j] + w[i][j] \times 1\]
  满足四边形不等式。
\end{frame}

\begin{frame}{Bridging signals}
  \framesubtitle{POJ 1631}
  最长上升子序列
\end{frame}

\begin{frame}{Longest Ordered Subsequence}
  \framesubtitle{POJ 2533}
  最长上升子序列
\end{frame}
\begin{frame}{Multiplication Puzzle}
  \framesubtitle{POJ 1651}
  经典区间动态规划,矩阵链乘。
  \[dp[i][j] = \min_{k=i}^{j-1}{dp[i][k] + dp[k+1][j] + row[i] \times col[k] \times col[j]}\]
\end{frame}

\begin{frame}{Brackets}
  \framesubtitle{POJ 2955}
  最大括号匹配数目。
  \begin{itemize}
    \item 当s[i] $=$ s[j]时,$dp[i][j] = \max{(dp[i+1][j-1], \max_{k=i}^{j-1}{(dp[i][k]+dp[k+1][j])})}$
    \item 当s[i] $\neq$ s[j]时,$dp[i][j] = \max_{k=i}^{j-1}{(dp[i][k]+dp[k+1][j])}$
  \end{itemize}
\end{frame}

\begin{frame}{Brackets}{Palindrome}
  \framesubtitle{POJ 1159}
  添加最少字符使字符串变成回文串。\\
  见上午课件
\end{frame}

\begin{frame}{玩具装箱}
  \framesubtitle{BZOJ 1010}
  设$a[i] = \sum_{j=1}^i{C[j]} + i$,可得
  \[\frac{(dp[k]+a[k]^2)−(dp[j]+a[j]^2)}{a[k]−a[j]} < 2a[i]−2(l+1)\]
\end{frame}

\begin{frame}{Sliding Window}
  \framesubtitle{POJ 2823}
  滑动窗口,用单调队列即可。
\end{frame}

\begin{frame}{Levels and Regions}
  \framesubtitle{Codeforces 674C}
  \[dp[i][j] = \min_{k=1}^{j-1}{dp[i-1][k] + cost(k+1, i)}\]
  怎么算$cost(k+1, i)$?
  \[cost(l,r)=\sum_{i=l}^{r}\frac{\sum_{j=l}^{i}{t_j}}{t_i}\]
  假设打$level_i$的概率是$p_i$,需要花费的期望时间是多少?
  \[1 + (1-p_i) + {(1-p_i)}^2 + \ldots = \frac{1}{p_i}\]
  原则:{\bf 独立性}。
\end{frame}

\begin{frame}{Sonya and Problem Wihtout a Legend}
  \framesubtitle{Codeforces 713C}
  先让$a[i]-=i$,变成一个单调不下降子序列的问题。\\
  然后我们发现,每次调整一个数,都使它变成一个序列中已经存在的数,这样会得到最优方案。\\
  所以dp状态可以设为:$dp[i][j]$表示第i个数变成了原序列中第j小的数的代价。
\end{frame}

\end{document}
