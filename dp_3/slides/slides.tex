\documentclass[hyperref={unicode=true}]{beamer}
\usepackage{multirow}
\usepackage{minted}
\usepackage{amsmath} % used for boldsymbol.
\renewcommand{\vec}[1]{\boldsymbol{#1}} % Uncomment for BOLD vectors.
\usepackage[slantfont,boldfont]{xeCJK}
\setCJKmainfont{SimSun}
\setCJKmonofont{SimHei}
\usetheme{Darmstadt}
\usecolortheme{beaver}
\usepackage{booktabs}

\input zhwinfonts
\begin{document}
\setbeamertemplate{caption}[numbered]
\renewcommand\figurename{图}
\renewcommand\tablename{表}
\renewcommand\contentsname{\centering 目录}

%%------------------------------------------
\title{动态规划进阶(三)}
\subtitle{再谈背包}
\author{马玉坤}
\institute{哈尔滨工业大学计算机科学与技术学院}
\date{2017年8月17日}
%%------------------------------------------

\begin{frame}\titlepage\end{frame}

\begin{frame}
  事情,要从一道背包题说起。。。
\end{frame}

\begin{frame}\frametitle{Clash Royale}
  \framesubtitle{Round A APAC Test 2017 D,China-Final 2016 热身赛C,2017黑龙江省赛I}
  \begin{block}{题目}
    有$n$张 ($n\leq12$)卡片,你现在需要将这$n$张卡片升级(也可以不升级),每张卡片可以升到的级别都小于等于10,每张卡片的每个级别都对应一个攻击力 ($\leq10^9$),每张卡片升级到一定级别都需要一定的金币 ($\leq10^9$)。\\
    给定金币数$C$,最大化$n$张卡片的攻击力之和。
  \end{block}
\end{frame}
\begin{frame}\frametitle{Clash Royale (Cont'd)}
  \framesubtitle{Round A APAC Test 2017 D,China-Final 2016 热身赛C,2017黑龙江省赛I}
  \begin{alertblock}{小小的尝试 (理论上)}
    令$f[i][j]$表示前$i$张卡片使用$j$块金币所能达到的最大攻击力和。\\
    \[f[i][j] = \max\begin{cases}
    f[i-1][j-cost[i][1]]+attack[i][1] &\\
    f[i-1][j-cost[i][2]]+attack[i][2] &\\
    \ldots &
    \end{cases}\]
  \end{alertblock}
\end{frame}
\begin{frame}[fragile]\frametitle{Clash Royale (Cont'd)}
  \framesubtitle{Round A APAC Test 2017 D,China-Final 2016 热身赛C,2017黑龙江省赛I}
  \begin{alertblock}{小小的尝试的改进}
    所有金币数(包括卡片升级需要的金币数)除以一个数,然后再对新的金币数进行动态规划。\\
    \begin{minted}{nasm}
      push 10
      push chr$("ACM/ICPC QingDao 2016 Online")
      push eip
      jmp  ANOTHER_PROBLEM
    \end{minted}
  \end{alertblock}
\end{frame}
\begin{frame}\frametitle{Herbs Gathering}
  \framesubtitle{ACM/ICPC Qingdao 2016 Online 10}
  \begin{block}{问题}
    一共有$n$ ($n\leq100$)株草药,采集每株草药都需要一定的时间 ($t_i\leq10^9$),采集每株草药都能获得一定的分数 ($score_i\leq10^9$),给定总时间$T$,问在$T$时间内能采集到的农药的总得分最大是多少。\\
    {\bf 数据随机生成}
  \end{block}
\end{frame}
\begin{frame}[fragile]\frametitle{Herbs Gathering (Cont'd)}
  \framesubtitle{ACM/ICPC Qingdao 2016 Online 10}
  \begin{exampleblock}{解法}
    第一印象:好大!\\
    第二印象:既然数据随机,所以是不是可以剪枝搜索?(可以过。)\\
    第三印象:既然数据随机,所以是不是可以强制减小采集草药的时间$t_i$和$T$?比如同除以$d=10^5$,这样$10^9$就变成了$10^4$,$10^9+7$也变成了$10^4$,如果WA就减小$d$,如果TLE就增大$d$。(也可以过。)
    \begin{minted}{nasm}
      pop  eip
    \end{minted}
  \end{exampleblock}
\end{frame}
\begin{frame}[fragile]\frametitle{Clash Royale (Cont'd)}
  \framesubtitle{Round A APAC Test 2017 D,China-Final 2016 热身赛C,2017黑龙江省赛I}
  \begin{alertblock}{小小的尝试的改进}
    所有金币数(包括卡片升级需要的金币数)除以一个数,然后再对新的金币数进行动态规划。\\
    \begin{minted}{nasm}
      push 10
      push chr$("ACM/ICPC QingDao 2016 Online")
      push eip
      jmp  ANOTHER_PROBLEM
    \end{minted}
    然而,这个方法只能过热身赛。(框的颜色已经剧透了。)
  \end{alertblock}
\end{frame}
\begin{frame}[fragile]\frametitle{Clash Royale (Cont'd)}
  \framesubtitle{Round A APAC Test 2017 D,China-Final 2016 热身赛C,2017黑龙江省赛I}
  回忆题目,12张牌,10种等级。$10^{12}$种方案。。。\\
  如果只有6张牌就好了。
  \pause\begin{exampleblock}{解法}
  \begin{enumerate}[1.]
  \item 枚举前六张牌的所有方案 ($10^6$),然后枚举后六张牌的所有方案 ($10^6$)。之后将前六张牌的所有方案按照需要的金币数排个序,后六张牌的所有方案按照所有的金币数排个序。\\
  \item 枚举前六张牌的方案,设当前枚举到的方案所需要的金币数为$c_1$,总攻击力为$a_1$,我们需要找到后六张牌的所有方案中需要金币数小于等于$C-c_1$的攻击力最大的方案。然后将这两个方案结合。
  \item 最优解一定在步骤2中产生。
  \end{enumerate}
  \end{exampleblock}
\end{frame}
\begin{frame}\frametitle{背包问题}
  \begin{block}{一点点理论}
    {\bf NP问题}:可在多项式时间内判断一个给定的解,对一个算法问题的实例,是否正确的问题。\\
    {\bf NP完全问题}:一类特殊的NP问题。若任何NPC问题得到多项式时间的解法,那此解法就可应用在所有NP问题上。\\
    {\bf 子集和问题}:一类NP完全问题。给一个整数集合和另一个整数$s$,问是否存在某个非空子集,使得子集中的数字和为$s$。\\
    同样,{\bf 背包问题}也属于NP完全问题。
  \end{block}
\end{frame}
\begin{frame}\frametitle{Divisible Subset}
  \framesubtitle{CodeChef DIVSUBS}
  \begin{block}{题目}
    给定$n$个数,找出一个子集,使得该子集中的数的和对$n$取模为$0$。
    $n\leq 10^5$
  \end{block}
\end{frame}
\begin{frame}\frametitle{Divisible Subset (Cont'd)}
  \framesubtitle{CodeChef DIVSUBS}
  \begin{alertblock}{尝试}
    类似于部分和问题。设$dp[i][j]$表示用前i个数,凑出的数对n取模为$j$是否可行。如果可行,$dp[i][j]=false$,否则$dp[i][j]=true$。那么我们有
    \[dp[i][j] = dp[i-1][(j-a[i])\%n] \vee dp[i-1][j]\]
    复杂度$O(n*n)$。
  \end{alertblock}
\end{frame}
\begin{frame}\frametitle{Divisible Subset (Cont'd)}
  \framesubtitle{CodeChef DIVSUBS}
  \begin{exampleblock}{解法}
    设序列前$i$个数的前缀和为$b_i$,那么我们就有$b_0,b_1,\ldots,b_n$等$n+1$个前缀和。这$n+1$个前缀和中,一定有两个对n的模数相同。\\
    设$b_l=b_r$,那么$\{a_{l+1},a_{l+2},\ldots,a_r\}$即为一个可行的子集。
  \end{exampleblock}
\end{frame}
\begin{frame}\frametitle{有限制的不定方程整数解问题}
  \begin{block}{问题}
    给定$a_1,\ldots,a_n$与$K$,求最小的$M$使得:
    \[a_1x_1+a_2x_2+ \cdots + a_n x_n=M\]有非负整数解,且
    \[M \ge K\]
    \[K\leq 10^{18},a_1a_2\ldots a_n\leq 10^{4n},n\leq 20\]
  \end{block}
\end{frame}
\begin{frame}[fragile]\frametitle{有限制的不定方程整数解问题 (Cont'd)}
  \begin{alertblock}{第一次尝试}
    {\bf 完全背包问题}:有若干种物品,每种物品都有无穷多个,而且每个都有相同的重量和价值(对于同一种物品来说)。要求在物品总重量小于等于$K$的前提下,物品总价值最大。\\
    完全背包问题可在$O(nK)$的时间复杂度内解决。\\
    \begin{minted}{C++}
      dp[0][0] = 0;
      for (int i = 0; i < n; i++) {
        for (int j = 0; j <= K; j++) {
          dp[i][j] |= dp[i-1][j];
          if (j >= a[i]) {
            dp[i][j] |= dp[i-1][j-a[i]] | dp[i][j-a[i]];
          }
        }
      }
    \end{minted}
  \end{alertblock}
\end{frame}
\begin{frame}[fragile]\frametitle{有限制的不定方程整数解问题 (Cont'd)}
  \begin{alertblock}{第二次尝试}
    我们把使式子$a_1x_1+a_2x_2+ \cdots + a_n x_n=M$有非负整数解的$M$称作合法的$M$。不失一般性地,设$a_1$为$a_1,\ldots,a_n$中最小的那个。我们把所有合法的$M$按对$a_1$取模后的模数分类。例如$n=2,a_1=5,a_2=7$,我们有:
    \begin{table}[H]
      \centering
      \caption{取模分类后的表}\label{table:div}
      \begin{tabular}{c@{}ccc}
        \toprule
        对5取模的结果 & \multicolumn{3}{|c}{合法的$M$}          \\ \midrule
        0       & \multicolumn{3}{|l}{0,5,10,\ldots}   \\
        1       & \multicolumn{3}{|l}{21,26,31,\ldots} \\
        2       & \multicolumn{3}{|l}{7,12,17,\ldots}  \\
        3       & \multicolumn{3}{|l}{28,33,38,\ldots} \\
        4       & \multicolumn{3}{|l}{14,19,24,\ldots}  \\ \bottomrule
      \end{tabular}
    \end{table}
  \end{alertblock}
\end{frame}
\begin{frame}[fragile]\frametitle{有限制的不定方程整数解问题 (Cont'd)}
  \begin{alertblock}{第二次尝试}
    每行中的相邻的合法的$M$都相差$5$。显然,如果$M$可行,那么$M+5$当然可行。\\
    如果我们对于每个模数,求出了最小的那个合法的$M$,问题就可以解决了。\\
    我们设$dp[i]$表示模数为$i$的合法的$M$的最小值,那么有:
    \[dp[i] = \min\begin{cases}
    dp[(i-a_1)\%a_1]+a_1 &\\
    dp[(i-a_2)\%a_1]+a_2 &\\
    \ldots &\\
    dp[(i-a_n)\%a_1]+a_n&
    \end{cases}\]
    等等,这根本不是动态规划,这不是最短路吗?
  \end{alertblock}
\end{frame}
\begin{frame}[fragile]\frametitle{有限制的不定方程整数解问题 (Cont'd)}
  \begin{exampleblock}{解法}
    建一个图,图中的节点编号分别为$0,1,\ldots,a_1-1$,点$i$有n条出边,第j条出边连向$(i+a_j)\%a_1$,边权为$a_j$,求$0$号点到其他各点的单元最短路。\\
    $0$号点到点$i$的最短路大小即为模数为$i$的合法的$M$中的最小值。
  \end{exampleblock}
\end{frame}
\begin{frame}\frametitle{The Sting}
  \framesubtitle{Yandex Algorithm 2017 Round1 D}
  \begin{block}{题目}
    一场球赛马上就要举行,对于你喜欢的球队来说,球赛有三种结果——赢、输或者平局。现在有n个人,他们都想跟你赌一把。\protect\footnote{本人不鼓励(平均情况下会赔钱的)赌博行为}\\
    每个人都有一个自己期望的比赛结果$r_i$与钱数$a_i$、$b_i$。你可以选择跟不跟他赌。如果比赛结果与他期望的比赛结果相同,你需要支付给这个人$a_i$,如果比赛结果是其他两种,那么这个人支付给你$b_i$。\\
    现在请你挑一些赌局,使得在最坏情况下得到的钱最多。\\
    {\bf 限制}:$1 \leq a_i, b_i, n \leq500$
  \end{block}
\end{frame}
\begin{frame}\frametitle{The Sting (Cont'd)}
  \framesubtitle{Yandex Algorithm 2017 Round1 D}
  \begin{exampleblock}{解法}
    赌局实际上可以分为三种:球队赢了赔钱的,输了赔钱的,平局了赔钱的。\\
    我们对赢钱换种理解方式:别人先给你$b_i$,如果你赌输了付给别人$a_i+b_i$。\\
    设你对于平局了赔钱类型的赌局选择的赌局集合为$S_0$。设赌赢了(也就是说比赛结果不是平局)得到的钱为$A_0$(实际上$A_0=\sum_{i\in S}b_i$),赌输了得到的钱(应该是负的)为$A_0-B_0$ ($B_0=\sum_{i\in S}a_i+b_i$)。\\
    同样的,对赢了赔钱的和输了赔钱的设对应的$S_1,A_1,B_1$和$S_2,A_2,B_2$。
  \end{exampleblock}
\end{frame}
\begin{frame}\frametitle{The Sting (Cont'd)}
  \framesubtitle{Yandex Algorithm 2017 Round1 D}
  \begin{exampleblock}{解法 (Cont'd)}
    假设我们选的赌局集合为$S_0 \cup S_1 \cup S_2$,那么我们最坏情况下得到的钱就有:
    \[M = A_0+A_1+A_2-\max{\{B_0,B_1,B_2\}}\]
    假设我们枚举$K=\max{\{B_0,B_1,B_2\}}$,那么问题就变成了三个独立的问题:
    \begin{itemize}
    \item 使$B_0\leq K$的的最大的$A_0$是多少?
    \item 使$B_1\leq K$的的最大的$A_1$是多少?
    \item 使$B_2\leq K$的的最大的$A_2$是多少?
    \end{itemize}
    这就变成了三个独立的0--1背包问题。
  \end{exampleblock}
\end{frame}
\end{document}
